% \documentclass[12pt]{article}

%%v helpful MIT exam schema, see http://www-math.mit.edu/~psh/exam/examdoc.pdf
 \documentclass[12pt]{exam} 
% \qformat{\textbf{Question \thequestion}\quad \thepoints\hfill } % Large depth to make space}
 \renewcommand{\thesubpart}{(\roman{subpart})}
 \renewcommand{\subpartlabel}{\thesubpart}
% \footer{}{}{Page \thepage\ de \numpages}
\addpoints
\pointsinmargin
% \boxedpoints
\renewcommand{\questionshook}{\setlength{\itemsep}{4mm}}
\renewcommand{\partshook}{
	\setlength{\itemsep}{0.2cm}
}

\usepackage{cmbright}

\pagestyle{empty} %for handouts
\addtolength{\jot}{2em} % this makes all formulae a bit more spaced vertically for handouts
\setlength{\parindent}{0pt} %we're not writing a novel
\renewcommand{\arraystretch}{1.5} %makes tables taller, so easier to write in

\usepackage[fleqn]{amsmath}
\usepackage{amssymb} %standard classy maths symbols
\usepackage[a4paper, margin=1in]{geometry} %makes it easy to specify where to put stuff on the page

\usepackage{siunitx} %v handy way of getting SI units to typeset correctly...
% \sisetup{locale = FR} %... and now it will use a comma for decimals

\usepackage[utf8]{inputenc}

% \usepackage[french]{babel} % frenchify everything (e.g. a Table becomes a Tableau)
% \usepackage{lmodern} % font with nicer French characters than standard
% \usepackage{textcomp} % and more help for special characters

\usepackage{tcolorbox} % basic but good package for boxes around text
\tcbset{colback=black!5!white}
% \usepackage{color} % colours
%\newcommand{\fillin}{\color{black!1!white}} %makes the text disappear
%\definecolor{fillin}{rgb} {1.00,1.00,1.00}
%\renewcommand{\fillin}{\color{blue}} \definecolor{fillin}{rgb} {0.00,0.00,1.00} %makes the filling in text appear (in blue)

\usepackage{enumitem} % can change the labels on lists, items, enumerates
\usepackage{setspace}

%\usepackage{tikz, pgfplots} % interval diagrams, sets, etc.
%\usetikzlibrary{calc,trees,positioning,arrows,fit,shapes}

\begin{document}

\textbf{Name:} \dotfill

\begin{questions}


\question[1] Un frère et une sœur sont nés un en été et un en hiver. \\
La sœur n'est pas née en hiver.
Qui est né en été ?

\fillwithdottedlines{7mm}

\question[4] Fill in the central column with the right symbol (or leave it blank).
\vspace{-4mm}
$$ \Rightarrow \qquad \Leftarrow \qquad \mathrm{or} \qquad  \Leftrightarrow$$
\vspace{-12mm}
\begin{center}
\begin{tabular}{|l|l|l|}
\hline
there are clouds           & \hbox to 3cm{} & it's raining              \\ \hline
I'm inside Rousseau         &  & I'm in Geneva      \\ \hline
I'm English             &  & I'm a teacher       \\ \hline
the glass is half full &  & the glass is half empty \\ \hline
\end{tabular}
\end{center}

\question[4] ``If two numbers are \textbf{even} \emph{(paires)}, then their \textbf{product} \emph{(produit)} is also even.''
\begin{parts}
\part Give an example for which this proposition is true.
\fillwithdottedlines{7mm}
\part Write down the hypothesis of this proposition.
\fillwithdottedlines{7mm}
\part Write down the conclusion of this proposition.
\fillwithdottedlines{7mm}
\part Prove this proposition.
\fillwithdottedlines{14mm}
\end{parts}

\question \textbf{Bonus.} Le professeur se rendait à son laboratoire en dehors de la ville à une vitesse constante (pas plus de 90 km/h).
À un moment donné, le compteur indiquait un kilométrage de 16961 km. Dans exactement 2 heures, le compteur affichait à nouveau un nombre qui se lit de la même façon dans les deux sens. \\
À quelle vitesse (km/h) le professeur roulait-il ? \hfill \emph{Travailler sur le verso.}

\end{questions}

\vfill

\begin{tcolorbox}

\textbf{Name of marker:}

\begin{center}
\gradetable[h][questions]
\end{center}

\end{tcolorbox}

\end{document}