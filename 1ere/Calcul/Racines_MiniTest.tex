% \documentclass[12pt]{article}

%%v helpful MIT exam schema, see http://www-math.mit.edu/~psh/exam/examdoc.pdf
 \documentclass[12pt]{exam} 
% \qformat{\textbf{Question \thequestion}\quad \thepoints\hfill } % Large depth to make space}
 \renewcommand{\thesubpart}{(\roman{subpart})}
 \renewcommand{\subpartlabel}{\thesubpart}
% \footer{}{}{Page \thepage\ de \numpages}
\addpoints
\pointsinmargin
% \boxedpoints
\renewcommand{\questionshook}{\setlength{\itemsep}{6mm}}
\renewcommand{\partshook}{
	\setlength{\itemsep}{0.2cm}
}

\usepackage{cmbright}

\pagestyle{empty} %for handouts
\addtolength{\jot}{2em} % this makes all formulae a bit more spaced vertically for handouts
\setlength{\parindent}{0pt} %we're not writing a novel
\renewcommand{\arraystretch}{1.5} %makes tables taller, so easier to write in

\usepackage[fleqn]{amsmath}
\usepackage{amssymb} %standard classy maths symbols
\usepackage[a4paper, margin=1in]{geometry} %makes it easy to specify where to put stuff on the page

\usepackage{siunitx} %v handy way of getting SI units to typeset correctly...
% \sisetup{locale = FR} %... and now it will use a comma for decimals

\usepackage[utf8]{inputenc}

% \usepackage[french]{babel} % frenchify everything (e.g. a Table becomes a Tableau)
% \usepackage{lmodern} % font with nicer French characters than standard
% \usepackage{textcomp} % and more help for special characters

\usepackage{tcolorbox} % basic but good package for boxes around text
\tcbset{colback=black!5!white}
% \usepackage{color} % colours
%\newcommand{\fillin}{\color{black!1!white}} %makes the text disappear
%\definecolor{fillin}{rgb} {1.00,1.00,1.00}
%\renewcommand{\fillin}{\color{blue}} \definecolor{fillin}{rgb} {0.00,0.00,1.00} %makes the filling in text appear (in blue)

\usepackage{enumitem} % can change the labels on lists, items, enumerates
\usepackage{setspace}

%\usepackage{tikz, pgfplots} % interval diagrams, sets, etc.
%\usetikzlibrary{calc,trees,positioning,arrows,fit,shapes}

% Use a 'closing' sqrt symbol
\usepackage{letltxmacro}
\makeatletter
\let\oldr@@t\r@@t
\def\r@@t#1#2{%
\setbox0=\hbox{$\oldr@@t#1{#2\,}$}\dimen0=\ht0
\advance\dimen0-0.2\ht0
\setbox2=\hbox{\vrule height\ht0 depth -\dimen0}%
{\box0\lower0.4pt\box2}}
\LetLtxMacro{\oldsqrt}{\sqrt}
\renewcommand*{\sqrt}[2][\ ]{\oldsqrt[#1]{#2}}
\makeatother

\begin{document}

\begin{minipage}{0.7\textwidth}
\textbf{Name:} \dotfill
\end{minipage}
\hfill
\textbf{No calculators}

\begin{questions}


\question[2] \textbf{Write down} \emph{(écrire)} the first 20 \textbf{squares} \emph{(carrés)} : $1; 4; 9; \ldots 200.$

\fillwithdottedlines{14mm}

\question[3] Let's compare $\sqrt{9+16}$ and $\sqrt{9} + \sqrt{16}$.

\begin{parts}
\part Calculate $\sqrt{9+16}$
\vspace{-3mm}
\uplevel{
\textbf{Attention !} \emph{La symbole de la racine sous-entend des paranthèses, ici} $\Big(\sqrt{(9+16)} \Big)$.
}
\fillwithdottedlines{8mm}

\part Calculate $\sqrt{9} + \sqrt{16} = $  \dotfill

\part True or false :  $\sqrt{9+16} = \sqrt{9} + \sqrt{16}$ ? \dotfill

\end{parts}

\question[3] Let's compare $\sqrt{4 \cdot 25}$ and $\sqrt{4} \cdot \sqrt{25}$.

\begin{parts}
\part Calculate $\sqrt{4\cdot25}= $  \dotfill

\part Calculate $\sqrt{4} \cdot \sqrt{25}= $  \dotfill

\part True or false $\sqrt{4\cdot25} = \sqrt{4} \cdot \sqrt{25}$ \dotfill
\end{parts}

\question[1] Which of these expressions are \textbf{always} \emph{(toujours)} true? Circle the correct rule.

$$ \sqrt{a + b} = \sqrt{a} + \sqrt{b} \hspace{15mm} \sqrt{a \cdot b} = \sqrt{a} \cdot \sqrt{b}$$

\question[1] \textbf{Find} \emph{(trouver)} the value of $\sqrt{6} \cdot \sqrt{10} \cdot \sqrt{15}= $  \dotfill
\fillwithdottedlines{13mm}

\vfill

\question \textbf{Bonus.} Find the value of $$ x = \sqrt{2+\sqrt {2+\sqrt { 2+\ldots}}}$$
\emph{Indice:} CRM p. 16. 
\hfill \textbf{Work on the back} \emph{(travailler sur le verso.)}


\end{questions}

\begin{tcolorbox}

\textbf{Name of marker:}

\begin{center}
\gradetable[h][questions]
\end{center}

\end{tcolorbox}

\end{document}