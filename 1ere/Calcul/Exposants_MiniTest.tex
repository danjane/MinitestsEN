% \documentclass[12pt]{article}

%%v helpful MIT exam schema, see http://www-math.mit.edu/~psh/exam/examdoc.pdf
 \documentclass[12pt]{exam} 
% \qformat{\textbf{Question \thequestion}\quad \thepoints\hfill } % Large depth to make space}
 \renewcommand{\thesubpart}{(\roman{subpart})}
 \renewcommand{\subpartlabel}{\thesubpart}
% \footer{}{}{Page \thepage\ de \numpages}
\addpoints
\pointsinmargin
% \boxedpoints
\renewcommand{\questionshook}{\setlength{\itemsep}{4mm}}
\renewcommand{\partshook}{
	\setlength{\itemsep}{0.2cm}
}

\usepackage{cmbright}

\pagestyle{empty} %for handouts
\addtolength{\jot}{2em} % this makes all formulae a bit more spaced vertically for handouts
\setlength{\parindent}{0pt} %we're not writing a novel
\renewcommand{\arraystretch}{1.5} %makes tables taller, so easier to write in

\usepackage[fleqn]{amsmath}
\usepackage{amssymb} %standard classy maths symbols
\usepackage[a4paper, margin=1in]{geometry} %makes it easy to specify where to put stuff on the page

\usepackage{siunitx} %v handy way of getting SI units to typeset correctly...
% \sisetup{locale = FR} %... and now it will use a comma for decimals

\usepackage[utf8]{inputenc}

\usepackage[french]{babel} % frenchify everything (e.g. a Table becomes a Tableau)
% \usepackage{lmodern} % font with nicer French characters than standard
% \usepackage{textcomp} % and more help for special characters

\usepackage{tcolorbox} % basic but good package for boxes around text
\tcbset{colback=black!5!white}
% \usepackage{color} % colours
%\newcommand{\fillin}{\color{black!1!white}} %makes the text disappear
%\definecolor{fillin}{rgb} {1.00,1.00,1.00}
%\renewcommand{\fillin}{\color{blue}} \definecolor{fillin}{rgb} {0.00,0.00,1.00} %makes the filling in text appear (in blue)

\usepackage{enumitem} % can change the labels on lists, items, enumerates
\usepackage{setspace}

%\usepackage{tikz, pgfplots} % interval diagrams, sets, etc.
%\usetikzlibrary{calc,trees,positioning,arrows,fit,shapes}

\begin{document}

\textbf{Nom:} \dotfill

\
\begin{questions}


\question[4] Remplir le tableau.

\begin{tabular}{|l|l|}
\hline
Expression avec multiplication & Expression avec exposant \\ \hline
$4 \cdot  4 \cdot 4 $                               &          $4^3$                 \\ \hline
 $2 \cdot 2 \cdot 2 \cdot 2 \cdot 2 \cdot 2$                           &                          \\ \hline
                               &     $x^5$                     \\ \hline
$(a + b) \cdot(a + b) \cdot(a + b)$                               &                          \\ \hline
\end{tabular}

\question[6] Remplir les valeurs.

\begin{tabular}{|l|l|}
\hline
Expression & Valeurs \\ \hline
$5^2$                               &       25                   \\ \hline
$25^0$                              &                          \\ \hline
$10^2$                              &                          \\ \hline
$2^{10}$                              &                          \\ \hline
$19^1$                              &                          \\ \hline
$3^4$                              &                          \\ \hline
$(3^2)^2$                              &                          \\ \hline
\end{tabular}

\question[2] Vrai ou faux ? $2^{100} < 1000^{10}$
\fillwithdottedlines{21mm}

\vfill

\question \textbf{Bonus.} Pour obtenir $8^8$, à quelle puissance faut-il élever $4^4$? \hfill \emph{Travailler sur le verso.}


\end{questions}

\begin{tcolorbox}

\textbf{Nom du correcteur:}

\begin{center}
\gradetable[h][questions]
\end{center}

\end{tcolorbox}

\end{document}