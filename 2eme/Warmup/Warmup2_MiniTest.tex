% \documentclass[12pt]{article}

%%v helpful MIT exam schema, see http://www-math.mit.edu/~psh/exam/examdoc.pdf
 \documentclass[8pt]{exam} 
% \qformat{\textbf{Question \thequestion}\quad \thepoints\hfill } % Large depth to make space}
 \renewcommand{\thesubpart}{(\roman{subpart})}
 \renewcommand{\subpartlabel}{\thesubpart}
 \footer{}{}{Page \thepage\ de \numpages}
\addpoints
% \pointsinmargin
% \boxedpoints
\renewcommand{\questionshook}{\setlength{\itemsep}{0.2cm}}
\renewcommand{\partshook}{
	\setlength{\itemsep}{0.1cm}
}

\usepackage{cmbright}

\pagestyle{empty} %for handouts
\addtolength{\jot}{2em} % this makes all formulae a bit more spaced vertically for handouts
\setlength{\parindent}{0pt} %we're not writing a novel
%\renewcommand{\arraystretch}{2} %makes tables taller, so easier to write in

\usepackage[fleqn]{amsmath}
\usepackage{amssymb} %standard classy maths symbols
\usepackage[a4paper, margin=1in]{geometry} %makes it easy to specify where to put stuff on the page

\usepackage{siunitx} %v handy way of getting SI units to typeset correctly...
% \sisetup{locale = FR} %... and now it will use a comma for decimals

\usepackage[utf8]{inputenc}

% \usepackage[french]{babel} % frenchify everything (e.g. a Table becomes a Tableau)
% \usepackage{lmodern} % font with nicer French characters than standard
% \usepackage{textcomp} % and more help for special characters

\usepackage{tcolorbox} % basic but good package for boxes around text
\tcbset{colback=black!5!white}
% \usepackage{color} % colours
%\newcommand{\fillin}{\color{black!1!white}} %makes the text disappear
%\definecolor{fillin}{rgb} {1.00,1.00,1.00}
%\renewcommand{\fillin}{\color{blue}} \definecolor{fillin}{rgb} {0.00,0.00,1.00} %makes the filling in text appear (in blue)

\usepackage{enumitem} % can change the labels on lists, items, enumerates
\usepackage{setspace}

\usepackage{tikz, pgfplots} % interval diagrams, sets, etc.
\usetikzlibrary{calc,trees,positioning,arrows,fit,shapes}

\newcommand\TikCircle[1][2.5]{\tikz[baseline=-#1]{\draw[thick](0,0)circle(3mm)[radius=#1mm];}}

\begin{document}

\textbf{Name:} \dotfill Consider the function $f( x ) = x^2 + 8x + 12$

\begin{questions}
\question[2] Complete the square and then use \emph{le troisième identité remarquable} to find the zeros of $f$. \\
\emph{Or for just one point, find the zeros with Viète:} $\frac{-b \pm \sqrt{b^2-4ac}}{2a}$
\fillwithdottedlines{5cm}

\question[4] Sketch the graph $y = f(x)$.

\vspace{-2mm}
\begin{minipage}[t]{0.5\textwidth}
\vspace{0mm}

\fillwithdottedlines{65mm}

\end{minipage}
\hfill
\begin{minipage}[t]{0.4\textwidth}
\vspace{0mm}

\begin{tikzpicture}[scale=0.4]
\draw[thick, ->] (-8,0) -- (8,0) node[above] {$x$};
\draw[thick, ->] (0,-8) -- (0,8) node[right] {$y$};


\end{tikzpicture}
\end{minipage}

\question[2] Solve the equation $(x+a)^2 = x^2+2x$, with $a\in \mathbb{R}$ a fixed parameter.
\fillwithdottedlines{45mm}

\question \textbf{Bonus.} To obtain $8^8$, to what power should one raise $4^4$?  \emph{Work overleaf.}

\vfill


\end{questions}

\begin{tcolorbox}

\textbf{Name of examiner:}

\begin{center}
\gradetable[h][questions]
\end{center}

\end{tcolorbox}

\end{document}